%!TEX root = main.tex
\chapter*{Resumen}
Verificar que un sistema de \emph{software} realiza correctamente las tareas para las cuales fue desarrollado es una de las actividades de mayor importancia en \emph{Ingenier\'ia de Software}, y concentra un significativo esfuerzo de investigaci\'on en esta \'area. El \emph{testing}, el cual consiste en ejecutar el programa a evaluar en un conjunto de escenarios particulares y contrastar el comportamiento esperado con el obtenido, es una de las t\'ecnicas m\'as utilizadas, como forma de comprobaci\'on del correcto comportamiento del software.

Dada la inherente incompletitud de \emph{testing}, una selecci\'on de los escenarios bajos los cuales realizar la evaluaci\'on, es necesaria. Claramente, c\'omo \'esta es realizada va a afectar la confianza que genere, cuando la ejecuci\'on real del software coincida con la esperada, es decir, que el conjunto seleccionado sea un buen representante de todos los posibles escenarios. Los criterios de testing permiten medir la calidad de un conjunto de tests generando objetivos a ser cubiertos, y evaluando cu\'antos de estos son satisfechos (cubiertos) por los tests. \emph{Mutation testing} es uno de estos, y consiste en inyectar fallas artificiales en el software bajo evaluaci\'on, y evaluar la capacidad de detecci\'on, por parte de los tests, de estas fallas.

Las fallas generadas por mutation testing se basan en operadores de mutaci\'on, las cuales deben ser buenos representantes de fallas reales, tradicionalmente generando cambios simples, tales como el reemplazo de operadores aritm\'eticos y relacionales. Estudios recientes han demostrado la falta de representaci\'on para ciertas fallas reales, incluso por operadores considerados \emph{suficientes}, principalmente en el contexto de programaci\'on orientada a objetos, usualmente obviada por los operadores actuales, motivando el desarrollo de nuevos operadores.

En esta tesis presentaremos un nuevo operador de mutaci\'on que aplica a expresiones de navegaci\'on, un tipo de expresiones muy utilizadas en programas orientados a objetos. Daremos una definici\'on formal del mismo, evaluando su aplicaci\'on en el contexto de \emph{mutation testing} y reparaci\'on de programas. Bajo principalmente colecciones implementadas en lenguajes orientados a objetos, mostraremos la producci\'on de fallas y generaci\'on de reparaciones no previamente representadas.

\noindent
\textbf{Palabras Clave:} Ingenier\'ia de Software, Mutaci\'on, Operadores de Mutaci\'on, Expresiones de navegaci\'on.






