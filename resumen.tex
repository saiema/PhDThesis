%!TEX root = main.tex
\chapter*{Resumen}

Verificar que un sistema de \emph{software} realiza correctamente las tareas para las cuales fue desarrollado es una de las actividades de mayor importancia en Ingenier\'ia de Software, y concentra un significativo esfuerzo de investigaci\'on en esta \'area. El \emph{testing}, el cual consiste en ejecutar un programa a evaluar en un conjunto de escenarios particulares y contrastar el comportamiento esperado del programa con el efectivamente obtenido, es una de las t\'ecnicas m\'as utilizadas como forma de comprobaci\'on del correcto comportamiento del software.

Dada la inherente incompletitud de \emph{testing}, resulta necesario realizar una selecci\'on adecuada de los escenarios bajos los cuales realizar la evaluaci\'on. Claramente, c\'omo esta evaluaci\'on es realizada va a afectar la confianza que genere el proceso de \emph{testing}, cuando la ejecuci\'on real del software coincida con la esperada, y las chances de detectar defectos en el \emph{software} durante este proceso. Concretamente, se espera que el conjunto de escenarios seleccionado para el proceso de \emph{testing} sea un buen representante de todos los posibles escenarios de ejecuci\'on del software en cuesti\'on. Los criterios de testing permiten medir la calidad de un conjunto de tests generando objetivos a ser cubiertos, y evaluando cu\'antos de estos son satisfechos (cubiertos) por los tests. \emph{Mutation testing} es uno de estos criterios, y consiste en inyectar fallas artificiales en el software bajo evaluaci\'on, y evaluar la capacidad de detecci\'on, por parte de los tests, de estas fallas.

Las fallas generadas por mutation testing se basan en operadores de mutaci\'on. Estos operadores deben ser buenos representantes de fallas reales, y tradicionalmente involucran cambios simples, tales como el reemplazo de operadores aritm\'eticos y relacionales. Algunos estudios recientes muestran sin embargo que algunas clases de fallas no se ven representadas por los operadores de mutaci\'on tradicionales, motivando as\'i la introducci\'on de nuevos operadores. 

En esta tesis presentaremos un nuevo operador de mutaci\'on, que aplica a \emph{expresiones de navegaci\'on}, un tipo de expresiones ampliamente utilizadas en programas orientados a objetos, y que no se ven afectadas por los operadores de mutaci\'on cl\'asicos. Daremos una definici\'on precisa del operador, y evaluaremos su aplicaci\'on tanto en el contexto de testing (mutation testing), como en el contexto de reparaci\'on de programas. 

\noindent
\textbf{Palabras Clave:} Ingenier\'ia de Software, Testing, Reparaci\'on de Programas, Mutaci\'on, Operadores de Mutaci\'on, Expresiones de Navegaci\'on.






