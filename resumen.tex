%!TEX root = main.tex
\chapter*{Resumen}
Verificar que un sistema de \emph{software} realiza correctamente las tareas para las cuales fue desarrollado es una de las actividades de mayor importancia en \emph{Ingenier\'ia de Software}, y concentra un significativo esfuerzo de investigaci\'on en esta \'area. El \emph{testing}, el cual consiste en ejecutar el programa a evaluar en un conjunto de escenarios particulares y contrastar el comportamiento esperado con el obtenido, es una de las t\'ecnicas m\'as utilizadas, como forma de comprobaci\'on del correcto comportamiento del software.

Dado que no es posible, en general, evaluar un programa en todos sus posibles escenarios de ejecuci\'on, el \emph{testing} involucra la selecci\'on de un conjunto finito de escenarios de ejecuci\'on, para realizar la evaluaci\'on. Claramente, c\'omo \'este es elegido va a afectar la confianza que genere, cuando la ejecuci\'on real del software coincida con la esperada, en la correcci\'on del software bajo evaluaci\'on. Los criterios de testing permiten medir la calidad de un conjunto de tests generando objetivos a ser cubiertos, y evaluando cu\'antos de estos son satisfechos (cubiertos) por los tests. \emph{Mutation testing} es uno de estos criterios de testing, y consiste en inyectar fallas artificiales en el software bajo evaluaci\'on, generando as\'i los objetivos a cubrir, y evaluando la capacidad de detecci\'on, por parte de los tests, de estas fallas.

Las fallas generadas por mutation testing se basan en operadores de mutaci\'on, donde cada operador encapsula una familia de cambios, tales como reemplazar operadores relacionales binarios en una expresi\'on por todos los otros disponibles en el lenguaje de programaci\'on correspondiente. Las fallas artificiales generadas por un operador deben estar limitadas en cantidad para mantener un rendimiento razonable, no deben ser triviales de detectar, no deben dar lugar a programas equivalentes, y deben representar fallas reales.

En esta tesis presentaremos una familia de operadores de mutaci\'on exclusivos para expresiones de navegaci\'on y el porqu\'e de la necesidad de \'estos. Evaluaremos su desempe\~no con respecto a criterios naturales de dise\~no para operadores de mutaci\'on, su utilizaci\'on tanto en el contexto de testing como de reparaci\'on autom\'atica de programas, y su rol en la generaci\'on autom\'atica de expresiones.

\noindent
\textbf{Palabras Clave:} Ingenier\'ia de Software, Mutaci\'on, Operadores de Mutaci\'on, Expresiones de navegaci\'on.






