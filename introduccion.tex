%!TEX root = main.tex
\chapter{Introducci\'on}
\label{cap:introduccion}
%Renzo, puso las siguientes cosas
%Reconocimiento de la importancia de la ing. de requisitos en metodologías de desarrollo (mencionar crisis del software). 

%Qué son los requisitos. 
%En qué consiste la captura, elaboración y especificación de requisitos.
%Por qué es importante?
%Por qué es arduo/complejo/costoso capturar requisitos?

%Metodológicamente, explicar en qué consiste tener una etapa de captura y especificación de requisitos concreta en el proceso de desarrollo.

%Etapa de análisis = captura y especificación de requisitos. Con qué contamos para hacerlo.Y captura? Varias herramientas vinculadas al “proceso”. Ej.: checklists, listas de preguntas frecuentes sobre el software a construir, etc. 

%Con qué contamos para hacerlo. Lenguaje natural, lenguajes de propósito específico (todo esto es especificación).
%Algunas herramientas metodológicas: Diagramas de estímulo/respuesta, DFDs y descomposición de funcionalidades, 4-variable model, KAOS. 

%natural 

%informales 

%Lenguaje natural vs. lenguajes formales para la especificación de requisitos. Sintaxis formal vs sintaxis+semántica. Posibilidades de análisis “objetivo” de specs, libre de interpretaciones subjetivas de las specs por parte de los ingenieros . Algunos antecedentes (mencionados superficialmente).

%Sintaxis + semántica formales facilitan análisis (semi-)automático vinculado a specs de requisitos. Qué se puede sistematizar? (algunos ejemplos de tareas).  Automático vs. asistido. Ejemplos de alto nivel, y mencionando sólo referencias, no trabajos concretos.
%Algunos ejemplos de trabajos explotando estas observaciones: referencias, descripciones superficiales.

%semi-automaticas 

%automaticas

% % % %


\section{Motivaci\'on y Objetivos}
\label{sec:intro.objetivos}
% RE etapa mas informal del proceso de desarrollo
% incertidumbre, desconocimiento, parcialidad.

%[operacional vs declarativo]

%Objetivos

\section{Estado del Arte}
\label{sec:intro.estado-del-arte}
%TODO: Describir brevemente diferentes tipos an\'alisis autom\'aticos que las herramientas y/o metodolog\'ias del estado-del-arte proveen. Mencionar fortalezas y debilidades.

\section{Contribuciones}
\label{sec:intro.contribuciones}
%TODO: Describir porque nuestras t\'ecnicas mejoran o complementan las t\'ecnicas de an\'alisis autom\'aticos existentes. Cuales y porque son los casos de estudios utilizados para validar cada una de las t\'ecnicas.
%\begin{itemize}
%\item operacionalizacion
%\item amplitud de objetivos (liveness)
%\item abstraccion
%\item escalabilidad
%\item herramientas
%\end{itemize}

%Se pueden mencionar papers publicados


\section{Organizaci\'on}
\label{sec:intro.organizacion}
%TODO: Cuales van a ser los temas de cada capitulo desarrollado en este trabajo.


