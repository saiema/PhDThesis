%!TEX root = main.tex
\chapter[Trabajo futuro]{Trabajo futuro}
\label{cap:futurework}

El trabajo presentado en esta tesis abre algunas l\'ineas de trabajo futuro. Por un lado, planeamos analizar m\'as profundamente cuales de las clases de fallas no acopladas a operadores actuales, identificadas en la secci\'on \ref{sec:prvo.prvoTargetedFaults}, est\'an acopladas a \emph{prvo}. Creemos que algunas est\'an directamente cubiertas por nuestro operador. Actualmente un estudio sobre este posible acoplamiento est\'a siendo realizado como parte del trabajo final de Licenciado en Ciencias de la Computaci\'on de Lea y del cual Pablo es director y yo co (REVISAR ESTO), \'este est\'a basado en el trabajo realizado por Ren\'e Just \cite{bibliography.mutation.evaluation.valid-substitute} e intenta encontrar si existe un acoplamiento de \emph{prvo} a fallas a\'un no acopladas en ese trabajo. Nuestro operador \emph{prvo} se presenta como un ``meta-operador'' altamente configurable, donde cada configuraci\'on se puede ver como un operador particular. La configuraci\'on del mismo se hace, en este trabajo, de una manera manual. Como automatizar esta configuraci\'on bas\'andose en caracter\'isticas del programa al cual est\'a siendo aplicado, forma parte de un camino de investigaci\'on a explorar. 

\section{PRVO en reparaci\'on autom\'atica de programas}

Como mencionamos en \ref{cap:repair}, reparaci\'on autom\'atica de programas utilizando mutaci\'on, es un \'area de investigaci\'on activa y en donde se debe balancear la capacidad de reparaci\'on, es decir, el conjunto de defectos que se pueden reparar y la cantidad de sentencias que pueden estar involucradas en la reparaci\'on, con los recursos necesario para encontrar a la misma. As\'i como nuestra motivaci\'on para proponer a \emph{prvo} como un operador a ser utilizado en \emph{mutation testing}, se basa principalmente en cuan extenso es el uso de expresiones de navegaci\'on en programas orientados a objetos, la misma puede aplicarse para motivar a \emph{prvo} como un operador a ser utilizado en la reparaci\'on de programas orientas a objetos. Junto a Frias, Luciano, Naza, etc, hemos trabajado en la herramienta llamada \emph{Striker}, presentada en \ref{sec:repair.striker}, la cual utiliza \emph{prvo} como uno de los operadores de mutaci\'on para la generaci\'on de candidatos a reparar un programa. Hemos observado numerosos casos en donde el uso de \emph{prvo} permite la reparaci\'on de fallas, que de otra forma, no podr\'ian ser encontradas. [AGREGAR CASOS EN QUE PRVO REPARA]