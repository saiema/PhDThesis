%!TEX root = main.tex
\chapter[PRVO]{PRVO una familia de operadores para expresiones de navegaci\'on}
\label{sec:prvo}

Para definir nuestra familia de operadores, primero necesitamos definir que consideramos como una \emph{expresi\'on encadenada}. Una expresi\'on encadenada involucra un \emph{operador de navegaci\'on}. En Lenguajes orientados a objetos, el operador de navefaci\'on puede tomar distintas formas sint\'acticas, con la notaci\'on punto siendo la encontrada com\'unmente. Su prop\'osito es acceder a miembros de una instancia de clase. Por simplicidad nos referiremos a una expresi\'on encadenada a una que cuenta con cero o m\'as de estos operadores. El tama\~no de una expresi\'on encadenada est\'a dada por el n\'umero de operadores de navegaci\'on involucrados. Una \emph{expresi\'on de navegaci\'on} es una expresi\'on encadenada de tama\~no 1 o m\'as. 

Una definici\'on directa de \emph{prvo} puede darse de la siguiente manera:
\begin{quote}
	Dada una expresi\'on encadenada $e$, \emph{prvo} va a generar mutaciones al reemplazar sub-expresiones en $e$, respectando el tipo de la expresi\'on, y manteniendo, incrementando, o decrementando su tama\~no.
\end{quote}
La Figura-\ref{figures.definitions.prvo.simple_def} define a la familia de operadores, como una gram\'atica. Como un ejemplo simple de mutaciones de \emph{prvo}, consideremos la expresi\'on \texttt{front}; las mutaciones generadas incluyen \texttt{front.next}, \texttt{null}, \texttt{front.next.next}, \texttt{front.next.elem}, \texttt{front.next.next.next}, entre otras. Un claro problema con la definici\'on anterior es que es recursiva y no acodata, en el sentido de que no hay un l\'imite sobre cuanto puede incrementarse o decrementarse el tama\~no de una expresi\'on.

\begin{figure}
	\begin{displaymath}
	\begin{array}{lll}
	PRVO(x)		& :=	& expression \\
	& := & PRVO(x).expression \\
	& := & expression.PRVO(x) \\
	\\
	PRVO(x.y)	& :=	& expression^{*} \\
	& :=	& PRVO(x).expression \\
	& :=	& expression.PRVO(y) \\
	& :=	& PRVO(x).expression.PRVO(y) \\
	& :=	& expression.PRVO(x).PRVO(y) \\
	& :=	& PRVO(x).PRVO(y).expression \\
	\\
	
	\multicolumn{3}{l}{\tiny{^{*} \: : \: puede \: incluir \: x \: or \: y}} \\
	\multicolumn{3}{l}{\tiny{expresi\'on \: : \: llamada \: a \: m\'etodo \: , \: acceso \: a \: atributo \: , variable \: \'o \: literal}}
	\end{array}
	\end{displaymath}
	\caption{Definici\'on abstracta de \emph{prvo}}
	\label{figures.definitions.prvo.simple_def}
\end{figure}

Teniendo en cuenta los criterios para el disen\~no de operadores de mutaci\'on, discutidos en [REFERENCIA], en particular con respecto al n\'umero de mutaciones que un operador produce, es necesario proveer algunas cotas razonables para la aplicaci\'on de \emph{prvo}. El n\'umero de mutantes generados por \emph{prvo} puede ser limitado al limitar tres caracter\'isticas: las expresiones \emph{objetivo} (donde se va a aplicar \emph{prvo}); el \emph{tama\~no de las expresiones}, por cuanto se permite cambiar, incrementar o reducir, el tama\~no de las expresiones resultantes (las mutaciones); y los \emph{reemplazos}, es decir, las expresiones que se van a usar para el intercalado/substituci\'on en \emph{prvo}. En cuanto a los objetivos, \emph{prvo} solo va a ser aplicado a expresiones de navegaci\'on, es decir, expresiones que involucran al menos una navegaci\'on. Respecto al tama\~no, vamos a limitar \emph{prvo} a producir expresiones del \emph{mismo} tama\~no que la expresi\'on original (el n\'umero de navegaciones se mantiene). En cuanto a los reemplazos, solo vamos a reemplazar expresiones con otras de exactamente el mismo tipo (al contrario de considerar definiciones menos estrictas, que permitir\'ian utilizar tipos compatibles m\'as generales), que pertenezcan a la misma clase en donde se encuentra la expresi\'on original, o en clases directamente alcanzables desde esta clase.

Claramente, estas limitaciones pueden ser pensadas como par\'ametros de \emph{prvo}, que en algunas situaciones pueden ser alteradas, permitiendo por un conjunto m\'as rico de expresiones a considerar en los reemplazos, permitiendo la generaci\'on de expresiones de navegaci\'on m\'as cortas/largas, etc. En nuestros caso, esta selecci\'on est\'a basada en experimentaci\'on, teniendo en cuenta los criterios para el disen\~o de operadores de mutaci\'on, mencionados anteriormente.