%!TEX root = main.tex
\chapter[Implementaci\'on]{PRVO y $\mu$Java++}
\label{sec:implementation}

\section{$\mu$Java++}

La herramienta \emph{$\mu$Java} \cite{bibliography.mutation.tools.muJavaMaOK05} funciona traduciendo c\'odigo fuente \emph{Java} a un \emph{AST}, una estructura de \'arbol que representa el c\'odigo original como nodos, y recorriendo el mismo mediante un patr\'on \emph{visitor}. Cada operador sobreescribe m\'etodos particulares de visita para los nodos objetivo, el operador \emph{ROR} solo tiene por objetivo a expresiones binarias (en cuanto al tipo de nodos) y particularmente a aquellas que utilicen un operador relacional. La versi\'on que fue desarrollada para poder implementar nuestra familia de operadores \emph{prvo}, contiene, a grandes rasgos, las siguientes caracter\'isticas que valen la pena mencionar:

\begin{enumerate}
	\item[Representaci\'on], las mutaciones generadas se representan mediante ternas \texttt{(original, mutante, operador)} en donde los dos primeros elementos representan nodos del \emph{AST} y el tercero denota el operador utilizado. \'Estas se almacenan durante la generaci\'on de mutaciones hasta que se decida generar los mutantes, esto permite analizar y evitar mutaciones repetidas.
	\item[Anotaciones], es posible utilizar comentarios \lstinline|//mutGenLimit K| luego de una sentencia para permitir que solo aquellas que tengan estas anotaciones y el \emph{K} sea mayor a 0, sean mutadas.
	\item[Control de generaciones], las anotaciones utilizadas son escritas en el mutante con el \emph{K} asociado disminu\'ido en 1. Esto permite, en casos como reparaci\'on o generaci\'on de mutantes compuestos (aquellos con m\'as de una mutaci\'on aplicada), controlar las generaciones m\'aximas por sentencia.
	\item[Estandarizaci\'on de componentes], la versi\'on original de la herramienta ten\'ia implementaci\'ones muy diferentes por operador, algunos escrib\'ian directamente cada mutantes (sin tener una representaci\'on para las mutaciones aplicadas), esto hac\'ia que desarrollar nuevos operadores fuera una tarea compleja, nuestra versi\'on permite desarrollar nuevos operadores con relativa facilidad.
	\item[Evaluaci\'on 2.0], mutation testing consiste en generar mutantes y evaluar cuantos son detectados por el conjunto de tests en evaluaci\'on. En la secci\'on \ref{sec:preliminares.mutation.opevaluation} se mencionan varias caracter\'isticas que afectan a este criterio, en particular dificultad de detecci\'on y redundancia de mutantes. Nuestra versi\'on de \emph{$\mu$Java} provee ciertas mejoras en el an\'alisis como ofrecer dos valores de \emph{mutation score}, uno en donde se consideran los mutantes que no compilaron y otro donde no se los considera; ofrecer un an\'alisis de ``dureza'' de los mutantes, al analizar cuantos tests fueron capaces de sobrevivir, no ser detectados por \'estos, antes de ser detectados; \emph{dynamic mutant subsumption}, en donde se genera un grafo de subsuma de mutantes para el test suite particular, permitiendo evaluar que mutantes fueron redundantes y cuales son indispensables; finalmente, para lidiar con la explosi\'on de mutantes que se puede dar y c\'omo el rendimiento es afectado, se implementaron la capacidad de frenar la ejecuci\'on de los tests para un mutante cuando \'este es detectado (perdiendo ciertos an\'alisis que requieren datos completos de la ejecuci\'on de los tests), y se introdujo el an\'alisis en paralelo de mutantes para acelerar el proceso.
\end{enumerate}

\section{PRVO}
\label{sec:implementation.prvo}

La definici\'on general de \emph{prvo}, dada en \ref{sec:prvo.prvo}, no es efectiva en la pr\'actica, desde ya por que permite entre otras cosas, infinitas mutaciones. Si bien se menciona que operadores de esta familia se definen mediante configuraciones particulares, en forma de restricciones, sobre la definici\'on general, la evoluci\'on de \emph{prvo} llev\'o a ciertas restricciones iniciales.

Inicialmente \emph{prvo} comienza como una necesidad en reparaci\'on autom\'atica de programas mediante mutaci\'on y \emph{SAT solving}. Utilizando la herramienta \emph{Stryker}, \emph{prvo} provee el potencial de reparar ciertos defectos no reparables anteriormente mediante mutantes. Como un problema que se menciona en \ref{sec:preliminares.repair}, la explosi\'on de candidatos en el espacio de b\'usqueda es uno de los principales obst\'aculos en reparaci\'on autom\'atica basada en mutaci\'on. Por esto, \emph{prvo} comienza con una serie de restricciones sobre el tama\~no de las expresiones mutadas, la diferencia entre el tama\~no original y final de una expresi\'on no puede superar a \texttt{1}. Estas restricciones llevan a una gram\'atica ligeramente distinta a la original.

\begin{figure}
	\begin{displaymath}
	\begin{array}{lll}
	PRVO(x)		& :=	& expression^{|1|} \\
	& := & x.expression^{|1-2|} \\
	& := & expression^{|1-2|}.x \\
	& := & null \\
	
	\\
	PRVO(x.y)	& :=	& x \\
	& :=	& 	y \\
	& :=	& expression^{|1-2|}.y \\
	& :=	& y.expression^{|1-2|} \\
	& :=	& expression^{|1-2|} \\
	& :=	& x.y.expression^{|1|} \\
	& :=	& x.expression^{|1|}.y \\
	& :=	& expression^{|1|}.x.y \\
	\\
	
	\end{array}
	\end{displaymath}
	\caption{Definici\'on base de la implementaci\'on de \emph{prvo}}
	\label{figures.definitions.prvo.impl_def}
\end{figure}

En la Figura-\ref{figures.definitions.prvo.impl_def} se restringe la definici\'on abstracta de \emph{prvo} [\ref{figures.definitions.prvo.simple_def}] en base a cuanto se puede cambiar la longitud de la expresi\'on original. La notaci\'on \emph{expression$^{|R|}$} denota una expresi\'on encadenada de un tama\~no particular o un rango de tama\~nos, por ejemplo, de 1 a 2.

\subsection{Configuraci\'on de PRVO}

Con el tiempo la cantidad de comportamientos o restricciones configurables en \emph{prvo} fue aumentando. En esta secci\'on solo mencionaremos un conjunto de \'estas, ya que varias fueron implementadas para casos muy particulares.

\subsubsection{Restricciones de tama\~no}

A partir de la definici\'on en la Figura-\ref{figures.definitions.prvo.impl_def}, las restricciones sobre cuanto puede decrementar o incrementar una expresi\'on encadenada se limitan a:

\begin{enumerate}
	\item[Reemplazar un elemento] Ciertas mutaciones de \emph{prvo} solo modifican elementos unitarios (expresiones de tama\~no 1) en una expresi\'on encadenada, reemplaz\'andolos por otros de tama\~no 1.
	\item[A\~nadir un elemento] Una expresi\'on encadenada puede incrementarse por 1 su tama\~no, al insertar un elemento tanto al principio, al final, o intercalarlo en su interior.
	\item[Eliminar un elemento] En una expresi\'on encadenada se puede eliminar un elemento, teniendo en cuenta que si la misma estaba solamente constitu\'ida por un elemento, entonces solo es posible decrementar la expresi\'on original reemplaz\'andola por \emph{null}.
	\item[Intercambiar dos elementos por uno] Dada una expresi\'on de navegaci\'on, dos elementos contiguos, es decir, una subexpresi\'on de navegaci\'on de tama\~no 1, puede ser reemplazada por un \'unico elemento.
	\item[Intercambiar un elemento por dos] En una expresi\'on encadenada, un elemento de la misma puede ser reemplazado por dos elementos contiguos, es decir, una expresi\'on de navegaci\'on de tama\~no 1.
\end{enumerate}

En el ejemplo siguiente pueden verse algunas de las distintas mutaciones que corresponden a las restricciones anteriores:
\begin{lstlisting}
  current = current.next;
  $\delta$ current = header;
  $\delta$ current = current;
  $\delta$ current = current.next.next;
  $\delta$ current = header.previous.next;
  previous = current;
  $\delta$ previous = header;
  $\delta$ previous = null;
  $\delta$ previous = current.next;
  $\delta$ previous = header.next;
\end{lstlisting}

\subsubsection{Restricciones de puntos de mutaci\'on}

Una expresi\'on encadenada como las que muta \emph{prvo} puede encontrarse en una gran cantidad de lugares distintas en el c\'odigo fuente. Si bien en el ejemplo anterior las expresiones mutadas fueron siempre sobre asignaciones y solo se aplicaban mutaciones a la parte derecha de la misma, \emph{prvo} cuenta actualmente con las siguientes restricciones configurables:

\begin{enumerate}
	\item[Parte izquierda de asignaciones], en una asignaci\'on \texttt{a = b}, \emph{prvo} puede o no generar mutaciones para \texttt{a}. Es necesario destacar que estas mutaciones no se aplican en declaraciones de variables, es decir, expresiones \texttt{T a = b} dado que modificar \texttt{a} en este caso genera en muchos casos mutantes que no compilan. Si bien es cierto que existen fallas en donde un desarrollador puede olvidar declarar una variable local y de esta forma hacer menci\'on a un atributo/campo de clase, este tipo de fallas no corresponde con las que se desea representar mediante \emph{prvo}.
	\item[Parte derecha de asignaciones], en una asignaci\'on \texttt{a = b}, \emph{prvo} puede o no generar mutaciones para \texttt{b}.
	\item[Sentencias de retorno y expresiones internas], \emph{prvo} es capaz de mutar expresiones encadenadas asociadas a sentencias de retorno (\texttt{return e}), al mismo tiempo que expresiones unarias y binarias encontradas en asignaciones \texttt{(a = e \&\& f)}, en sentencias condicionales \texttt{(while(c) ...)}, argumentos de m\'etodos \texttt{foo(x)}, e incluso en las distintas partes de sentencias \texttt{for} como la inicializaci\'on, condici\'on e incremento.
\end{enumerate}

\subsubsection{Restricciones de expresiones a utilizar}

Incluso luego de configurar restricciones sobre cuanto disminuir o incrementar la expresi\'on encadenada original, y en que partes del c\'odigo aplicar \emph{prvo}. La cantidad de expresiones v\'alidas disponibles para mutar una expresi\'on encadenada, sigue siendo demasiado grande. Esto trae aparejado problemas de eficiencia (la cantidad de mutantes afecta los recursos necesarios para mutation testing), generaci\'on de mutantes triviales o con poca dificultad para ser detectados, y mutantes redundantes.

La expresi\'on \lstinline|Object obj = current.next;|, puede dar lugar a una enorme cantidad de mutaciones v\'alidas, dado que no solo cualquier tipo no primitivo hereda de \texttt{Object}, sino que adem\'as, \emph{Java} permite ``autoboxing'' de tipos primitivos. Esta t\'ecnica permite que expresiones como \lstinline|Integer i = 1;| sean v\'alidas e internamente se traducen a \lstinline|Integer i = new Integer(1);|. Volviendo al ejemplo anterior podemos ver que es posible generar una gran cantidad de mutaciones completamente v\'alidas desde un punto de vista de tipos, pero completamente independientes de los tipos que pueden ser de inter\'es o que est\'an involucrados en la expresi\'on original. Ejemplos de \'estas son:
\begin{lstlisting}
  Object obj = current.toString();
  Object obj = current.getClass();
  Object obj = current.hashCode();
  Object obj = current.toString().length;
  Object obj = current.next.toString();
  ...
\end{lstlisting}

Por esto, \emph{prvo} cuenta con las siguientes restricciones configurables:

\begin{enumerate}
	\item[M\'etodos y campos restringidos], es posible definir expresiones regulares para restringir el uso de ciertos m\'etodos y campos, por ejemplo \lstinline|java\\.lang\\.String\\#.*| restringe cualquier m\'etodo y campo de \emph{java.lang.String}. Durante la generaci\'on de mutaciones, \emph{prvo} verifica que cualquier m\'etodo o campo que vaya a utilizar para mutar una expresi\'on encadenada, no est\'e restringida.
	\item[M\'etodos y campos permitidos], en muchos casos, la cantidad de m\'etodos y campos a restringir es muy grande y solo se quieren permitir aquellos que pertenezcan a ciertas clases, por ejemplo, en el caso de una lista puede solo desearse utilizar aquellos que pertenezcan a la clase \emph{Lista} y \emph{Nodo}. Es posible definir, de la misma forma que en el caso anterior, que m\'etodos y campos son permitidos para que \emph{prvo} los utilice durante la generaci\'on de mutaciones. Cabe destacar que estas opciones son exclusivas, no es posible utilizar ambas al mismo tiempo.
	\item[Control de tipos], en lenguajes orientados a objetos como \emph{Java}, la herencia de clases permite que dos tipos sean compatibles a\'un cuando no son iguales. Dados dos tipos, \texttt{A} y \texttt{B}, tal que el segundo hereda del primero. La asignaci\'on \lstinline|A a = new B();| es v\'alida mientras que la inversa no lo es. Una expresi\'on encadenada mutada por \emph{prvo} tiene que ser correcta con respecto a tipos, pero con esto podemos cambiar expresiones como \lstinline|list.node.value| a \lstinline|list.toString().getClass()| si \texttt{value} era de tipo \texttt{Object}. Por esto, \emph{prvo} puede ser restringirse a utilizar un control estricto de tipos, en donde \'estos solo se consideran compatibles sin son exactamente iguales.
	\item[Uso de literales], en ciertos casos el uso de literales conforman expresiones v\'alidas para mutar una expresi\'on encadenada, \lstinline|var.toString().length| puede cambiarse a \lstinline|"".length|. Cualquier operador que utilice valores literales, va a requerir un conjunto finito de los mismos. En el caso de \emph{prvo}, utiliza un conjunto de literales base, \emph{1,0,True,False,"",null} en donde cada uno puede habilitarse o no; a su vez permite la b\'usqueda de literales alcanzables desde el punto en donde se est\'a mutando; tambi\'en es posible habilitar/deshabilitar la generaci\'on de variaciones en literales num\'ericos, es decir que para cada literal que pertenezca a un tipo primitivo num\'erico, se van a crear copias del mismo para los otros tipos. Por ejemplo, para \texttt{2}, un literal de tipo \texttt{int} se van a crear las variantes \texttt{2.0d}, \texttt{2.0f}, y \texttt{2l}.
	\item[Uso de campos est\'aticos], el modificador \emph{static} en \emph{Java} define a un miembro que es compartido por todas las instancias de la clase que lo define. Estos miembros est\'aticos no deber\'ian accederse mediante instancias de una clase, aunque hacerlo es posible y no representa c\'odigo inv\'alido. Sin embargo si \'esto se permite, se pueden obtener muchas mutaciones asociadas al uso de constantes est\'aticas que no fueron restringidas mediante otra configuraci\'on. Por eso, \emph{prvo} permite la restricci\'on del uso de miembros est\'aticos en un contexto no est\'atico.
\end{enumerate}

\section{Dynamic mutant subsumption}
\label{sec:implementation.dynamicSubsumption}

El uso de \emph{mutation score}, la relaci\'on de mutantes detectados sobre los totales, como m\'etrica de evaluaci\'on para test suites, trae varios problemas sobre que conclusiones se pueden derivar del mismo. Entre \'estas, hemos mencionado la dificultad de detectar mutantes incrementando el valor de mutation score sin implicar una mejora en la capacidad de detectar fallas del test suite; la existencia de mutantes que son sem\'anticamente equivalentes al programa original y por lo tanto imposibles de detectar, bajando el valor del mutation score sin significar una menor capacidad de detectar fallas; finalmente el acoplamiento, es decir, que tan bien los mutantes representan fallas reales, y los mutantes redundantes, hacen que el an\'alisis de \emph{mutation testing} requiera cuidado al interpretar los resultados, y estudios m\'as profundos para dar mayor validez a los mismos. Estos estudios a\~naden complejidad al an\'alisis y, en la mayor\'ia de los casos, requiere resolver problemas indecidibles o altamente complejos. \texttt{Mutant Subsumption} representa un an\'alisis que eval\'ua la capacidad de evaluar a los tests de parte de los mutantes.
\begin{quote}
	Para dos mutantes \emph{m$_1$} y \emph{m$_2$}, producidos a partir del programa original \emph{p}. Se define la relaci\'on \emph{m$_1$} subsume a \emph{m$_2$}, si existe al menos un test para el cual el comportamiento de \emph{m$_1$} difiera del de \emph{p}, y para todo test \emph{t} para el cual el comportamiento entre \emph{m$_1$} y \emph{p} difiera, el comportamiento entre \emph{m$_2$} y \emph{p} tambi\'en debe hacerlo.
\end{quote}
Un punto importante es que la definici\'on anterior habla del universo de tests, todo test posible, lo cual es un problema que es indecidible. Antes de pasar a la implementaci\'on de este an\'alisis, es necesario mostrar cual es la utilidad del mismo.
La intuici\'on es que si un mutante es detectado por una gran cantidad de tests mientras otros son detectados por subconjuntos que detectan al primero, entonces detectar cualquiera de \'estos lleva a detectar tambi\'en al primero, evaluando varias veces los mismos tests. Por otro lado, un mutante que requiere tests m\'as espec\'ificos para ser detectado genera un mejor ``feedback'' sobre los tests que uno que es detectado por casi cualquiera. Como mencionamos, en la pr\'actica no es posible hacer un an\'alisis basado en la definici\'on anterior, por lo que la siguiente definici\'on es utilizada.
\begin{quote}
	Para dos mutantes \emph{m$_1$} y \emph{m$_2$}, producidos a partir del programa original \emph{p}, y un conjunto de tests \emph{T}. Se define la relaci\'on \emph{m$_1$} subsume din\'amicamente a \emph{m$_2$}, si existe al menos un test en \emph{T} para el cual el comportamiento de \emph{m$_1$} difiera del de \emph{p}, y para todo test \emph{t} en \emph{T} para el cual el comportamiento entre \emph{m$_1$} y \emph{p} difiera, el comportamiento entre \emph{m$_2$} y \emph{p} tambi\'en debe hacerlo.
\end{quote}
Esta definici\'on da a lugar a \emph{Dynamic Mutant Subsumption}, y es la que se implement\'o en \emph{$\mu$Java++} y que va a ser utilizada como evaluaci\'on adicional, en este caso para el conjunto de mutantes generados en cada experimento.

\subsection{Mutantes dominadores}
\label{sec:implementation.dynamicSubsumption.dominators}