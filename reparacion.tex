\chapter{Mutaci\'on de expresiones de navegaci\'on en reparaci\'on de programas}
\label{cap:repair}

\section{Reparaci\'on autom\'atica de programas}
\label{sec:repair}

La aparici\'on de t\'ecnicas autom\'aticas para testing y localizaci\'on de fallas, es decir, detectar en donde puede estar una falla, una vez detectada, llev\'o a la investigaci\'on de t\'ecnicas para, una vez detectada, poder reparar dicha falla de manera autom\'atica. Pero que significa reparar una falla?
\begin{quote}
	Dado un programa \textbf{P} y una especificaci\'on \textbf{E} del mismo, si \textbf{P} no cumple con sus especificaciones (\textbf{E}), reparar al programa significa encontrar un variante \textbf{P'} mediante modificaciones sint\'acticas tal que \'esta cumpla con las especificaciones dadas por \textbf{E} \footnote{Esta definici\'on no considera reparaciones en tiempo de ejecuci\'on, es decir, aquellas que no producen una variante del programa original como reparaci\'on}.
\end{quote}
A partir del trabajo de \cite{bibliography.repair.ArcuriY08}, muchas herramientas que intentan atacar a este problema han surgido. Incluso cuando la idea de reparaci\'on autom\'atica de fallas es atractiva, reparar autom\'aticamente defectos de programas arbitrarios es inviable. Por lo tanto, la reparaci\'on autom\'atica de programas debe sacrificar completitud. Varias t\'ecnicas efectivas para reparaci\'on de programas recurren a explorar un conjunto grande (aunque limitado) de candidatos de reparaci\'on obtenidos por modificaciones sint\'acticas de un programa con fallas. Adem\'as, para que estas t\'ecnicas escalen razonzablemente, el espacio de candidatos a reparaci\'on debe ser acotado, limitando el conjunto de modificaciones sint\'acticas a considerar (por ejemplo, no considerar modificaciones a partes de una sentencia), o no explorando exhaustivamente el conjunto acotado de candidatos a reparaci\'on (por ejemplo, usando algor\'itmos gen\'eticos en lugar de b\'usqueda exhaustiva).

\subsection{Or\'aculo de reparaci\'on}
\label{sec:repair.specs}

En general el proceso de reparaci\'on sigue, a rasgos muy generales, el siguiente algoritmo:

\begin{lstlisting}
  repair(P, E) {
    current = P
    while (!isValid(current, E) && !boundsReached()) {
      current = nextFix();
    }
    return current;
  }
\end{lstlisting}

Donde \textbf{P} representa el programa a reparar, y \textbf{E} las especificaciones que se deben satisfacer. El proceso b\'usca candidatos a reparaciones, evaluando a \'estos con respecto a las especificaciones provistas. Claramente el proceso est\'a acotado a un conjunto finito de posibles reparaciones.

Evidentemente, es necesario definir que se usa como especificaciones para el programa. Los primeros trabajos sobre reparaci\'on autom\'atica \cite{bibliography.repair.StaberJB05, bibliography.repair.ArcuriY08}, utilizaban especificaciones formales en forma de pre y post condiciones, o descripciones l\'ogicas provistas en alg\'un formalismo l\'ogico apropiado. Sin embargo, muchos de las \'ultimas t\'ecnicas utilizan tests como especificaciones. La raz\'on principal detr\'as de esto, es el argumento de que la producci\'on de especificaciones formales es un proceso costoso, a la vez que es muy raro encontrarlas ya provistas previo al proceso de reparaci\'on, mientras que los tests son menos costosos de proveer y es mucho m\'as com\'un encontrarlos ya provistos antes del proceso de reparaci\'on.

Por ejemplo, \emph{GenProg} \cite{bibliography.repair.GouesNFW12} usa computaci\'on evolutiva para evolucionar sint\'acticamente un programa hasta que una reparaci\'on aceptable es encontrada. Cada reparaci\'on candidata (modificaci\'on sint\'actica) es aplicada al programa original para producir uno nuevo cuya aptitud es evaluada utilizando un test suite. Modificaciones sint\'acticas en partes de una sentencia no son consideradas para limitar el espacio de candidatos, y la funci\'on de aptitud es utilizada para mantener una poblaci\'on reducida de candidatos a lo largo del proceso de computaci\'on evolutiva. 

\subsubsection{Tests como especificaciones}
\label{sec:repair.specs.testsAsSpecs}

Las herramientas actuales, en general, utilizan tests como especificaciones parciales del programa a reparar. Esto, tiene como consecuencia la generaci\'on de reparaciones espurias, es decir, reparaciones que si bien logran hacer ``pasar'' a los tests, no resuelven el problema. En muchos casos, enmascarar un defecto puede causar que los tests que antes detectaban una falla, ahora dejen de hacerlo. A modo de ejemplo, en la Figura \ref{figures.examples.repair.nullaccessexample} se muestra un m\'etodo \emph{find(int)} que tiene un defecto, la condici\'on del \textbf{while} deber\'ia incluir que \textbf{current} no sea \emph{null} y que el elemento a\'un no fue encontrado. Sin embargo, la reparaci\'on en la Figura \ref{figures.examples.repair.nullaccessexample.mask} enmascara al defecto. Si nuestros tests solo consta de listas en donde el elemento no est\'a, y por lo tanto se espera que el m\'etodo retorne \emph{false}; o el elemento est\'a en el \'ultimo elemento, para los casos donde se espera que retorne \emph{true}. La reparaci\'on anterior es espuria, es decir, no repara realmente el programa.

\begin{figure}
	\begin{lstlisting}[mathescape=true, language=Java]
  public boolean find(int e) {
    if (isEmpty()) return false;
    Node current = header;
    boolean found = false;
    while(true) {
      found = current.elem == e;
      current = current.next;
    }
    return found;
  }
	\end{lstlisting}
	\caption{Un m\'etodo con un defecto que causa \emph{NullPointerException}}
	\label{figures.examples.repair.nullaccessexample}
\end{figure}

\begin{figure}
	\begin{lstlisting}[mathescape=true, language=Java]
  public boolean find(int e) {
    if (isEmpty()) return false;
    Node current = header;
    boolean found = false;
    while(true) {
      try {
        found = current.elem == e;
        current = current.next;
      } catch (NullPointerException e) {
        break;
      }
    }
    return found;
  }
	\end{lstlisting}
	\caption{Reparaci\'on que enmascara el defecto en \ref{figures.examples.repair.nullaccessexample}}
	\label{figures.examples.repair.nullaccessexample.mask}
\end{figure}

Esta observaci\'on llev\'o a un trabajo de investigaci\'on en donde se evaluaron un conjunto de herramientas actuales de reparaci\'on autom\'atica de programas basadas en tests como or\'aculo de reparaci\'on \cite{bibliography.repair.ZeminBGCDRAF17}. Al contrario de trabajos previos que se basaron en tests o inspecci\'on manual de las reparaciones provistas por las herramientas evaluadas (\emph{GenProg}, \emph{Angelix}, \emph{AutoFix}, y \emph{Nopol}, entre otras), en \'este, se utilizaron especificaciones formales y \emph{PEX} \cite{bibliography.formalver.TillmannH08} para evaluar las reparaciones obtenidas.

Los resultados obtenidos, logrados sobre el benchmark de reparaci\'on de programas \emph{IntroClass} \cite{bibliography.repair.benchmarks.GouesHSBDFW15}, indican que al evaluar las reparaciones producidas utilizando especificaciones formales, \'estas son inv\'alidas, es decir, el programa obtenido no cumple con las especificaciones del problema que deben resolver. El porcentaje de reparaciones para \emph{GenProg}, que son consideradas como efectivas, es solo un 1.57\% (18 de 1.143 fallas fueron reparadas) de las cuales 8 reparan errores de typos en cadenas asociadas a los mensajes de salida del programa. \emph{Angelix} solo es capaz de reparar 41 de 232 (17,67\%), \emph{Nopol} solo 6 de 297 (2\%), y finalmente \emph{AutoFix} no pudo reparar ning\'un programa de manera correcta\footnote{Algunas herramientas no soportaban ciertas caracter\'isticas (retornar una cadena), y en otras que requer\'ian traducir a otro lenguaje, la misma reparaba el defecto}. Por el otro lado, incrementar la cantidad de tests disminu\'ia dr\'asticamente las reparaciones espurias, pero tambi\'en las reparaciones en general. En otros casos, las herramientas no soportaban el aumento en la cantidad de tests.

\subsection{Granularidad de las reparaciones}
\label{sec:repair.granularity}

Como mencionamos anteriormente, el conjunto de candidatos a reparaci\'on a tener en cuenta, debe ser finito y su tama\~no afecta directamente a los recursos necesarios para reparar el programa. Esto, genera una necesidad de balancear, capacidad de reparaci\'on, es decir, que tipo de fallas se pueden reparar, con recursos necesarios. Por ejemplo, el caso de \emph{GenProg}, se basa en mover, duplicar, o eliminar c\'odigo. La idea es que muchas veces un desarrollador escribe varias veces el mismo c\'odigo, por ejemplo, sumarle 1 a una variable. Por otro lado, en \emph{SPR} \cite{bibliography.repair.LongR15}, se generan reparaci\'ones abstractas como agregar una condici\'on \textbf{C} antes de la ejecuci\'on de una sentencia, la cual luego ser\'a reemplazada por una condici\'on concreta en una etapa posterior, esto sumado a la eliminaci\'on, y duplicaci\'on de c\'odigo existente. En \emph{PAR} \cite{bibliography.repair.KimNSK13}, las modificaciones para reparar el programa son aprendidas a partir de patrones en reparaciones escritas manualmente. Por esto, el n\'umero de candidatos a considerar como reparaciones es significativamente reducido, lo que a su vez reduce el tipo de errores que la t\'ecnica puede ser capaz de reparar.

Modificaciones a partes de sentencias, es decir, aquellas que alteran expresiones dentro de una sentencia, son en general no consideradas por t\'ecnicas de reparaci\'on de programas. Una limitaci\'on principal al considerar a \'estas, es la explosi\'on en el espacio de candidatos de reparaci\'on. T\'ecnicas que utilizan operadores de mutaci\'on para producir las modificaciones sint\'acticas, y que incluyen modificaciones a partes de sentencias, requieren limitar el conjunto de mutaciones (por ejemplo, \cite{bibliography.repair.GopinathMK11}), reduciendo la clase de fallas que \'estas pueden intentar reparar.

\begin{figure}
\begin{lstlisting}[mathescape=true, language=Java,basicstyle=\footnotesize]
boolean add(Object arg) {
  SListNode freshNode = new SListNode();
  freshNode.value = arg;
  boolean added = false;
  if (this.header == null) {
    this.header = freshNode;
    added = true;
  } else {
    SListNode current = this.header.next; //BUG: ignora primer nodo
    while (current.next ! = null && current.value ! = arg) {
      current = current.next;
    }
    if (current.value ! = arg) {
      current.next = freshNode;
      added = true;
    }
  }
  if (added) {
    size = size - 1; //BUG: decrementa size
  }
  return added;
}
\end{lstlisting}
\caption{Ejemplo de c\'odigo con dos bugs que requieren modificaciones intra-sentencia}
\label{figures.examples.repair.exampleCode}
\end{figure}

Los efectos que tienen estas restricciones sobre el espacio de candidatos a considerar para la reparaci\'on, pueden observarse en la Figura \ref{figures.examples.repair.exampleCode} donde se muestra el m\'etodo \emph{add(Object)} de un conjunto. \'Este contiene dos defectos: primero, cuando el conjunto no est\'a vaci\'o, se comienza el recorrido del mismo a partir del nodo siguiente al inicial (el cual podr\'ia ser \emph{null}, resultando en un error en la l\'inea siguiente); segundo, cuando un nuevo nodo es agregado a la lista sobre la cual est\'a implementado el conjunto, el tama\~no de la misma (atributo \emph{size}) se decrementa. Para reparar al m\'etodo es necesario realizar dos cambios, ambos dentro de una sentencia. Si bien ser\'ia posible conseguir el c\'odigo correcto en otro lado del programa, \textbf{SListNode current = this.header;} y \textbf{size = size + 1} respectivamente.


\subsubsection{Mutaci\'on en reparaci\'on}
\label{sec:repair.granularity.mutation}

El uso de mutaci\'on en reparaci\'on de programas parece razonable, si se utiliza mutaci\'on para emular defectos reales, podr\'ia utilizarse para emular reparaciones. Herramientas de reparaci\'on basada en mutaci\'on existen, \cite{bibliography.repair.mutation.DebroyW10, bibliography.repair.mutation.AlloyWang18} son ejemplos recientes. El argumento de reparaci\'on basada en mutaci\'on, es que existen mutaciones que si se fueran a combinar, se cancelar\'ian, por ejemplo:
\begin{lstlisting}[mathescape=true]
  for (int i = 0; i < lenght; i++)...
  $\delta$for (int i = 0; i > lenght; i++)...
\end{lstlisting}
donde la primera sentencia se puede mutar, aplicando un cambio de operador relacional, a la segunda, marcada por $\delta$, que a su vez se puede mutar al c\'odigo original aplicando el mismo operador de mutaci\'on. Aunque no siempre se puede deshacer una mutaci\'on aplicando otra que sea sint\'acticamente inversa. 
Volviendo al ejemplo anterior, el mutante:
\begin{lstlisting}[mathescape=true]
  for (int i = 0; i > lenght; i++)...
\end{lstlisting}
puede ser restaurado, sem\'anticamente, generando el mutante:
\begin{lstlisting}[mathescape=true]
  for (int i = 0; i $\textbf{!=}$ lenght; i++)...
\end{lstlisting}
Existen tambi\'en casos donde varias mutaciones pueden corregir el comportamiento. Esto lleva a la idea de que si consideramos el programa con fallas \texttt{P$_b$} y el original sin fallas \texttt{P$_o$}, se puede definir el segundo en t\'erminos del primero como \texttt{P$_b$ = mutate(P$_o$, M)} donde \texttt{M} representa una secuencia de mutaciones y \texttt{mutate} es un programa que aplica dicha secuencia a un programa. En general como dijimos anteriormente, toda mutaci\'on tiene su inversa, lo que lleva a definir el problema de reparaci\'on como encontrar una secuencia de mutaciones \texttt{M$\prime$} tal que \texttt{P$_o$ = mutate(P$_b$, M$\prime$)}.

La reparaci\'on de programas mediante mutaci\'on puede describirse como una b\'usqueda exhaustiva que, dado un programa defectuoso y una especificaci\'on del mismo (que como vimos puede ser formal o mediante tests), y un conjunto de operadores de mutaci\'on:
\begin{enumerate}
	\item Considera el programa a reparar como el candidato a reparaci\'on inicial.
	
	\item Si \textbf{P} es un candidato a reparaci\'on, y \textbf{Q} es el resultado de aplicar una mutaci\'on a \textbf{P}, entonces \textbf{Q} tambi\'en es un candidato a reparaci\'on.
	
	\item  Un candidato \textbf{S} es exitoso si satisface las especificaciones provistas.
\end{enumerate}

\begin{figure}
	\begin{tikzpicture}%
	[state/.style ={ellipse, draw, minimum width = 0.7 cm},
	point/.style = {circle, draw, inner sep=0.04cm,fill,node contents={}},
	el/.style = {inner sep=2pt, align=left, sloped}]
	\node[state,rectangle] (init) {$return\:a < b?a:b;$};
	\node[state,rectangle] (cand1) [below=of init] {$return\:b < b?a:b;$};
	\node[state,rectangle] (cand2) [right=of cand1] {$return\:a < b?a:0;$};
	\node[state,rectangle] (succ) [below=of cand1] {$return\:b < a?a:b;$};
	
	\path[->] (init) edge node[left] {a -> b} (cand1);
	\path[->] (init) edge node[right=16pt] {b -> 0} (cand2);
	\path[->] (cand1) edge node[left] {b -> a} (succ);
	
	\node[label=right: {\small Inicial}, draw=blue,dotted,fit=(init)] (initR) {};
	
	\node[label=right: {\small Candidatos (Profundidad 1)},draw=blue,dotted,fit=(cand1) (cand2), inner sep=0.2cm] (Candidates) {};
	
	\node[label=right: {\small Exitoso (Profundidad 2)},draw=blue,dotted,fit=(succ)] (Succ) {};
	
	\end{tikzpicture}
	\caption{Reparaci\'on del un programa que calcula el m\'aximo de dos n\'umeros}
	\label{figures.examples.repairWithMutation}
\end{figure}

La definici\'on previa de reparaci\'on basada en mutaci\'on deja en claro que el espacio de candidatos a reparaci\'on depende del n\'umero de mutaciones (\textbf{b}) a considerar, lo que influye en cuan ``ancho'' es el \'arbol de b\'usqueda, y el n\'umero m\'aximo de mutaciones sucesivas permitidas (\textbf{d}) para generar a los candidatos, lo que afecta la profundidad de la b\'usqueda. El espacio de b\'usqueda queda entonces definido por una suma geom\'etrica $\frac{b^{d+1}-1}{b-1}$ ($O(b^d)$). Si consideramos el m\'etodo en la Figura \ref{figures.examples.repair.getNode}, y solo cuatro l\'ineas del mismo, al utilizar 18 operadores de mutaci\'on (mediante la herramienta \emph{$\mu$Java++}), podemos ver en la Tabla \ref{tables.repair.mutation.explosion}, como a medida que aumentamos la cantidad de mutaciones consecutivas (la profundidad en reparaci\'on mediante mutaci\'on), la cantidad de candidatos a evaluar se vuelve r\'apidamente inmanejable.

\begin{figure}
	\begin{lstlisting}[mathescape=true,language=Java,numbers=left,xleftmargin=20pt]
  Node getNode(int i) {
    Node current = this.head;
    Node result = null;
    int current_index = 0;
    while (result == null && current != null) {
      if (i == current_index ) {
        result = current;
      }
      current_index = current_index + 1;
      current = current.next;
    }
    return result;
  }
	\end{lstlisting}
	\caption{C\'odigo de ejemplo, m\'etodo que obtiene el i-\'esimo nodo de una lista}
	\label{figures.examples.repair.getNode}
\end{figure}

\begin{table}[t]
	\begin{center}
		\small
		\begin{tabular}{c r}
			Search Depth                            &	No. of Mutants (Fix Candidates)        \\
			\hline
			1 					&	40		                               	\\
			2 					&	1,604			                \\
			3 					&	64,684		        	        \\
			4 					&	$>$ 20 million		                
		\end{tabular}
		\normalsize
	\end{center}
	\caption{Mutantes generados para \emph{getNode} \ref{figures.examples.repair.getNode}, considerando 4 l\'ineas y 18 operadores, a medida que la profundidad aumenta.}
	\label{tables.repair.mutation.explosion}
\end{table}


\subsection{Striker}
\label{sec:repair.striker}

\emph{Striker} es una herramienta de reparaci\'on autom\'atica de programas cuya meta principal es:
\begin{quote}
	Proveer una t\'ecnica automatizada y eficiente para reparar programas anotados con especificaciones (en t\'erminos de pre y post condiciones) corrigiendo errores que resultan como una consecuencia de ocurrencias simultaneas de un n\'umero de errores sint\'acticos dentro de sentencias de un programa.
\end{quote}
Es necesario remarcar la necesidad de especificaciones asociadas al programa. As\'i tambi\'en, los errores que se consideran son defectos, o mutaciones, dentro de sentencias. En particular no se apunta a reparar programas que requieran agregar c\'odigo. Algunas de los defectos reparables por \emph{Striker} se muestran en la Figura \ref{figures.examples.repair.faultsGetNode}.

Si bien muchas herramientas de reparaci\'on autom\'atica de programas utilizan mutaci\'on, \emph{Striker} es una de las m\'as flexibles con respecto a operadores soportados, entre ellos \emph{prvo}; soporte para modificaciones a partes de sentencias; y fallas que requieren las modificaci\'on de varias sentencias. Striker utiliza \emph{TACO} \cite{bibliography.mutation.tools.TACOGaleottiRPF13} para evaluar candidatos a reparaci\'on, lo que requiere proveer especificaciones formales en lugar de un test suite, y JML RAC \cite{bibliography.misc.JMLRAC.LeavensCCRC02} como t\'ecnica de evaluaci\'on r\'apida basada en escenarios previamente encontrados para los cuales un candidato a reparaci\'on no cumpl\'ia con las especificaciones. Sin entrar en detalles, esta herramienta es capaz de expandir las fallas soportadas para reparar, agregando varios operadores en los que incluye a \emph{prvo}, y permitir la reparaci\'on de aquellas que requieren modificaciones intra-sentencias as\'i como en m\'ultiples sentencias, mediante el uso de t\'ecnicas de poda innovadoras.

Bajo esta herramienta es que se pudo observar como \emph{prvo} era capaz de reparar ciertas fallas que de otra forma no eran reparables, como por ejemplo el caso de \emph{add(int)} en la Figura \ref{figures.examples.repair.exampleCode}, en donde si bien el segundo defecto es reparable por \emph{AORB} el cual reemplaza el operador incorrecto (\emph{-}) por el que corresponde (\emph{+}), el primer defecto solo puede ser reparado por \emph{prvo} al eliminar el acceso al campo \emph{next}. Si bien el enfoque principal de esta tesis se encuentra en \emph{prvo} como un operador de mutaci\'on en el contexto de mutation testing, es interesante mostrar su contribuci\'on en el contexto de reparaci\'on.

\begin{figure}[t]
	\footnotesize
	$$
	\begin{array}{rl}
	5_a: & (\mathit{result} \mbox{ != }\mathsf{null}\ \&\&\ \mathit{current}\mbox{ != }\mathsf{null})\\
	5_b: & (\mathit{result} == \mathsf{null}\ \&\&\ \mathit{current} == \mathsf{null})\\
	5_c: & (\mathit{result} == \mathsf{null}\ ||\ \mathit{current}\mbox{ != }\mathsf{null})\\
	6_a: & (\mathit{i} == \mathit{current\_index} + 1)\\
	6_b: & (\mathit{i}\ \mbox{!=}\ \mathit{current\_index})\\
	9_a: & \mathit{current\_index} = \mathit{current\_index} - 1\\
	10_a: & \mathit{current.next} = \mathit{current}
	\end{array}
	$$
	\normalsize
	\caption{Algunos defectos en el m\'etodo \emph{getNode} \ref{figures.examples.repair.getNode} que pueden modelarse mediante mutaciones y ser reparados por \emph{prvo}.}
	\label{figures.examples.repair.faultsGetNode}
\end{figure}

\subsection{Evaluaci\'on de Stryker}
\label{sec:repair.striker.evaluation}

Si bien la evaluaci\'on de \emph{Stryker} se encuentra enfocada hacia la t\'ecnica de poda del espacio de b\'usqueda de candidatos a reparaci\'on, lo que a su vez permite la utilizaci\'on de una mayor cantidad de operadores de mutaci\'on orientados hacia la reparaci\'on de fallas intra-sentencia, en el contexto de esta tesis nos interesan los resultados asociados a \emph{prvo} y su utilidad en el campo de reparaci\'on autom\'atica de programas mediante mutaci\'on.

En los experimentos realizados para la evaluaci\'on de \emph{Stryker}, se utilizaron implementaciones en \emph{Java} de estructuras que representan colecciones. Estas clases, para las cuales se definieron especificaciones en forma de contratos \emph{JML}, incluyendo clausulas \texttt{requires/ensures}, funciones de variantes de ciclos e invariantes de clases, son las siguientes:
\begin{itemize}
	\item \textbf{SinglyLinkedList :} Una implementaci\'on de listas simplemente encadenadas. Donde consideramos un m\'etodo \emph{contains} para evaluar pertenencia, \emph{getNode} para obtener el i-\'esimo elemento en la lista, y el m\'etodo \emph{insert} para agregar un nuevo elemento a la lista.
	
	\item \textbf{NodeCachingLinkedList :} Una lista circular, doblemente encadenada y con una cache de nodos. Considerando los m\'etodos \emph{contains}, \emph{inserte}, y \emph{remove}. Siendo el \'ultimo de particular inter\'es por almacenar nodos removidos en la cache.
	
	\item \textbf{BinarySearchTree :} Una implementaci\'on de un \'arbol de b\'usqueda binario con los m\'etodos \emph{contains}, \emph{insert} y \emph{remove}.
	
	\item \textbf{BinomialHeap :} Una implementaci\'on de colas de prioridad utilizando \emph{binomial heaps}. Considerando los m\'etodos \emph{findMin} (para obtener el m\'inimo elemento almacenado), \emph{insert}, y \emph{extractMin} para obtener y eliminar el m\'inimo elemento almacenado.
\end{itemize}

Los casos de estudio para reparaci\'on fueron obtenidos a partir de las versiones originales (correctamente implementadas) al insertar hasta $4$ mutaciones por m\'etodo, y luego se eligieron 5 versiones al azar por cada n\'umero de mutaciones insertadas, es decir, 5 programas incorrectos con un bug, 5 con 2, y as\'i hasta $4$ bugs.

Por cada sentencia (o l\'inea) mutada se agrego un comentario de l\'inea \texttt{//mutGenLimit K} donde \texttt{K} corresponde a la cantidad de bugs artificiales introducidos en la misma. Algunos ejemplos de los casos de estudios utilizados se muestran a continuaci\'on.

\begin{figure}
	\lstinputlisting[basicstyle=\footnotesize, language=Java, tabsize=3]{results/repair/BinomialHeap_extractMin_orig.java}
	\caption{Versi\'on original del m\'etodo \emph{extractMin} de \emph{BinomialHeap} con los contratos asociados al m\'etodo.}
	\label{figures.code.repair.binheap_extractMin_orig}
\end{figure}

\begin{figure}
	\lstinputlisting[basicstyle=\footnotesize, language=Java, tabsize=3]{results/repair/BinomialHeap_extractMin_4bugs.java}
	\caption{Versi\'on con $4$ bugs artificiales del m\'etodo \emph{extractMin} de \emph{BinomialHeap}.}
	\label{figures.code.repair.binheap_extractMin_4bugs}
\end{figure}

\begin{figure}
	\lstinputlisting[basicstyle=\footnotesize, language=Java, tabsize=3]{results/repair/BinTree_insert_orig.java}
	\caption{Versi\'on original del m\'etodo \emph{insert} de \emph{BinarySearchTree} con los contratos asociados al m\'etodo.}
	\label{figures.code.repair.bintree_insert_orig}
\end{figure}

\begin{figure}
	\lstinputlisting[basicstyle=\footnotesize, language=Java, tabsize=3]{results/repair/BinTree_insert_3bugs.java}
	\caption{Versi\'on con $3$ bugs artificiales del m\'etodo \emph{insert} de \emph{BinarySearchTree}.}
	\label{figures.code.repair.bintree_insert_3bugs}
\end{figure}

\begin{figure}
	\lstinputlisting[basicstyle=\footnotesize, language=Java, tabsize=3]{results/repair/NCLL_contains_orig.java}
	\caption{Versi\'on original del m\'etodo \emph{contains} de \emph{NodeCachingLinkedList} con los contratos asociados al m\'etodo.}
	\label{figures.code.repair.ncll_contains_orig}
\end{figure}

\begin{figure}
	\lstinputlisting[basicstyle=\footnotesize, language=Java, tabsize=3]{results/repair/NCLL_contains_4bugs.java}
	\caption{Versi\'on con $4$ bugs artificiales del m\'etodo \emph{contains} de \emph{NodeCachingLinkedList}.}
	\label{figures.code.repair.ncll_contains_4bugs}
\end{figure}