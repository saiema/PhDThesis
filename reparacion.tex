\chapter[Reparaci\'on]{Reparaci\'on autom\'atica de programas}
\label{sec:preliminares.repair}

En reparaci\'on autom\'atica de programas, uno de los argumentos m\'as utilizados es que por un lado, cuando existe una falla, la versi\'on del programa sin la misma es muy cercana, sint\'acticamente, a la primera. Por otro lado, el c\'odigo asociado a la reparaci\'on es, en muchos casos, una reorganizaci\'on (eliminaci\'on, desplazamiento, o duplicaci\'on) de c\'odigo existente; una alteraci\'on, es decir, un cambio sint\'actico, de c\'odigo existente; o una combinaci\'on de ambas. Dentro de las t\'ecnicas de reparaci\'on asociadas mayormente a la primer categor\'ia se encuentran [AGREGAR]. La segunda categor\'ia est\'a muchas veces asociada a el uso de mutation dentro del contexto de reparaci\'on, el argumento es que existen mutaciones que si se fueran a combinar, se cancelar\'ian, por ejemplo:
\lstinline|for (int i = 0; i < lenght; i++)...| se puede mutar, aplicando un cambio de operador relacional, a \lstinline|for (int i = 0; i > lenght; i++)...|, que a su vez se puede mutar al c\'odigo original aplicando el mismo operador de mutaci\'on. Aunque no siempre se puede deshacer una mutaci\'on aplicando otra que sea sint\'acticamente inversa. Volviendo al ejemplo anterior, el mutante \lstinline|for (int i = 0; i > lenght; i++)...| se puede restaurar, sem\'anticamente, generando el mutante \lstinline|for (int i = 0; i != lenght; i++)...|. Existen tambi\'en casos donde varias mutaciones pueden corregir el comportamiento. Esto lleva a la idea de que si consideramos el programa con fallas \texttt{P$_b$} y el original sin fallas \texttt{P$_o$}, se puede definir el segundo en t\'erminos del primero como \texttt{P$_b$ = mutate(P$_o$, M)} donde \texttt{M} representa una secuencia de mutaciones y \texttt{mutate} es un programa que aplica dicha secuencia a un programa. En general como dijimos anteriormente, toda mutaci\'on tiene su inversa, lo que lleva a definir el problema de reparaci\'on como encontrar una secuencia de mutaciones \texttt{M$\prime$} tal que \texttt{P$_o$ = mutate(P$_b$, M$\prime$)}. Herramientas que tratan este caso, el de reparar programas utilizando mutaci\'on o una combinaci\'on entre mutaci\'on y reorganizaci\'on de c\'odigo incluyen [AGREGAR]

\section{Striker}
\label{sec:preliminares.repair.striker}